\documentclass[10pt]{siamltex}
% siamltex     final
\usepackage{amsmath,amssymb,epsfig}
%\usepackage[]{graphicx,epsf,epsfig,amsmath,amssymb,amsfonts,latexsym,amsthm}
% \usepackage{bbm}
%\usepackage{amsfonts}
%\usepackage{amsthm}
%% \usepackage[a4paper]{geometry}
%% \geometry{top=0.99in, bottom=0.995in, left=0.995in, right=0.995in}
% \usepackage[top=20mm, bottom=18mm, left=15mm, right=18mm]{geometry}
\usepackage[top=18mm, bottom=18mm, left=20mm, right=20mm]{geometry}


\usepackage{hyperref}

% \graphicspath{{/Users/Mihai/Google Drive/MATLAB/Ranking_Sync/PLOTS/}}
% D:/LATEX/Reports@IIT/figures/
\usepackage{graphicx}
\usepackage[space]{grffile}

\newcommand{\mb}[1]{\mbox{\boldmath$#1$}}

\usepackage{lineno}
\pagewiselinenumbers

\usepackage{subfigure}
\usepackage{amsmath}
\usepackage{amssymb}
\usepackage{graphicx}
\usepackage{comment}
\usepackage{array}
\usepackage{algorithm}
\usepackage{algorithmic}
\usepackage{url}
% \usepackage{color}

% \usepackage[toc,page]{appendix}

%\usepackage[outer]{showlabels,rotating}
%\renewcommand{\showlabelsetlabel}[1]
%{\begin{turn}{0}\showlabelfont #1\end{turn}}
%\showlabels{cite}
%\showlabels{ref}
%\showlabels{foo}
%\renewcommand{\showlabelfont}{\small}

\usepackage{setspace}
\usepackage{multicol}
\usepackage{multirow}
\usepackage{color}
\usepackage{colortbl}
\usepackage{xcolor}
\usepackage{hyperref}
%\def\Acronimo{ANALYSISOFNETWORKS}
% number the lines

\newcommand{\myreferences}{../../../Postdocs_laptop/postdoc_bib_file}
% \newcommand{\myreferences}{/Users/Mihai/test_bib_file}

\usepackage{lineno}
% \pagewiselinenumbers
%\setlength\linenumbersep{8pt}

% ANALYSISOFNETWORKS
% AALGN
\hypersetup{
%    pdftitle={\Acronimo{}, FP7-PEOPLE-IIF-2012},    % title
%    pdfauthor={Some Author},
    colorlinks=true,
    citecolor=red,
    linkcolor=blue,
    urlcolor=blue
  }

%
%\newtheorem{theorem}{Theorem}[section]
%\newtheorem{conjecture}[theorem]{Conjecture}
%\newtheorem{corollary}[theorem]{Corollary}
%\newtheorem{proposition}[theorem]{Proposition}
%\newtheorem{lemma}[theorem]{Lemma}
%\newdef{definition}[theorem]{Definition}
%\newdef{remark}[theorem]{Remark}
\newcounter{ale}
\newcommand{\abc}{\item[\alph{ale})]\stepcounter{ale}}
\newenvironment{liste}{\begin{itemize}}{\end{itemize}}
\newcommand{\aliste}{\begin{liste} \setcounter{ale}{1}}
\newcommand{\zliste}{\end{liste}}
\newenvironment{abcliste}{\aliste}{\zliste}

% \usepackage[english]{datetime}



\begin{document}


\begin{pagewiselinenumbers}
\title{MATH 191 Graphs and Networks \\ Research Project}
\author{First~Author \footnotemark[1], Second~Author \footnotemark[2], Third~Author \footnotemark[1]  }
\date{ \today}
% \vspace{-10mm}
\maketitle
%\renewcommand{\thefootnote}{\fnsymbol{footnote}}
\footnotetext[1]{Department of Mathematics, UCLA, 520 Portola Plaza, Mathematical Sciences Building 6363, Los Angeles, CA 90095-1555, email: xyz@ucla.edu}
\footnotetext[2]{Second Department, UCLA, 520 Portola Plaza, Mathematical Sciences Building 6363, Los Angeles, CA 90095-1555}
%\renewcommand{\thefootnote}{\arabic{footnote}}

\begin{center}
     February 28, 2015
%  DRAFT
\end{center}

% We propose novel approach to the
\vspace{5mm}

\begin{abstract}
Abstract goes here. Provide 10-15 lines with a summary of your work.
\end{abstract}


\begin{keywords} A few keyword that best describe your work. For example:  network centrality, spectral algorithms, gravity models, vertex similarity, core periphery structure, directed graphs, local-to-global algorithms, multiple network alignment, clustering bipartite graphs, angular synchronization
\end{keywords}


\section{Introduction}

This is the Introduction.

\vspace{10mm}  % add white space whenever needed

This is how you cite a reference paper  \cite{corePerOxford} or multiple ones  \cite{corePerOxford,puckmason}.




\begin{figure}[h]
\begin{center}
\includegraphics[width=0.46\columnwidth]{Directed_Core_Per}
\end{center}
\caption{This is how you add a plot to a Figure. It is always a good idea to add such a caption to each Figure, and explain what the figure is about. Also, if your plot has x and y axis, please always label your graphs (within MATLAB) so that your plots/results can be easily read and understood.}
\label{fig:DirCorPer}
\end{figure}


This is how you cite a (labeled) section: In Section \ref{sec:relWork}, we describe the related work considered in \cite{puckmason}.
%%
In Section \ref{sec:ourWork} we ... 
In Section  \ref{sec:NumExpSyn} we test the above algorithms on synthetically generated data sets, while in Section \ref{sec:NumExpReal} we do so for a real data set. Finally, in Section  \ref{sec:conclusion} we conclude with a summary of our results, and discuss future possible research direction. 



\section{Related work} \label{sec:relWork}

\textbf{Questions/Comments/Things that could be done}
\begin{itemize}
\item What is a good notion of core-periphery structure in directed networks? Would the null model shown in Figure \ref{fig:DirCorPer} be a good model?
\item Can one build on or expand some of the above methods and apply them to directed networks? 
\item As a starting point, perhaps apply simulated annealing to the objective function induced by the above null model
\item Apply this to a real network, a good such example might be the migration network between counties in the United States, which we have seen in class in the past.
\item This is how you add an url link 
 \begin{center}
   \url{http://people.maths.ox.ac.uk/porterm/papers/prestige_final.pdf} 
\end{center}   
\end{itemize}

\textcolor{red}{This is how you color tex in red ...}

\textcolor{blue}{This is how you color tex in blue ...}

\section{Our work}    \label{sec:ourWork}
Describe the bulk of your work in this section.

\section{Numerical experiments on synthetic data}  \label{sec:NumExpSyn}
Present here the numerical results you obtain on a synthetically generated data set...

\section{Numerical experiments on real data}   \label{sec:NumExpReal}
Present here the numerical results you obtain on a real data set...


\section{Summary and conclusion}  \label{sec:conclusion}
Summarize your work in this section.


\bibliographystyle{siam}
\bibliography{project_math191}

\end{pagewiselinenumbers}

\end{document}



