\documentclass{beamer}
\usecolortheme{dolphin}
%\usetheme{Warsaw}

\usepackage{amsmath,amssymb,amsrefs}
\usepackage{amscd}
\usepackage{latexsym}
\usepackage{graphics}
\usepackage{subfigure}
\usepackage{placeins}
\usepackage{booktabs}
\usepackage{multirow}
\usepackage{times}  % fonts are up to you
\usepackage[authoryear]{natbib}
\def\newblock{\hskip .11em plus .33em minus .07em} %this line is used to
% fix the bug of natbib that ``\newblock'' undefined.
\usepackage{array}

\usepackage{lipsum}

 \setbeamersize{text margin left=11pt,text margin right=8pt}



\newcommand\mydef{\mathrel{\overset{\makebox[0pt]{\mbox{\normalfont\tiny\sffamily def}}}{=}}}

\usepackage{soul}



\newif\ifpaper

% \papertrue % or
\paperfalse


\usepackage{graphicx}
\usepackage{nicefrac}
\usepackage{lscape}
\usepackage{times}
% \usepackage[dvips,colorlinks=true,linkcolor=red,filecolor=green,citecolor=red]{hyperref}
%\usepackage[]{amsmath,amssymb,epsfig}
% \usepackage{subfigure}
\usepackage{mathtools}%
\usepackage{algorithm}
\usepackage{algorithmic}
\usepackage{wrapfig}





\newcommand{\mb}[1]{\mbox{\boldmath$#1$}}

\setbeamertemplate{headline}{\scriptsize{\vspace*{0.1cm}\hspace*{0.1cm}\insertframenumber}}

\title{Discovering Hidden Prerequisites}
\author{ Anumat Srivastava, Lowell Bander, Paul Jayazeri\\
\vspace{5mm}
MATH 191 Topics in Data Science, UCLA}
% \institute[]{ Applied Mathematics, UCLA
% Applied and Computational Mathematics, \\
% Princeton University
% }
%\address{PACM Graduate Student Seminar}
% \date{ Harvard University \\ February 24, 2014}

\date{ \today}

% note: do NOT include a \maketitle line; also note that this title
% material goes BEFORE the \begin{document}

%%%%% have this if you'd like a recurring outline
\AtBeginSection[]  % "Beamer, do the following at the start of every section"
{
\begin{frame}<beamer>
  \frametitle{Outline} % make a frame titled "Outline"
  \tableofcontents[currentsection]  % show TOC and highlight current section
\end{frame}
}


\begin{document}

\newtheorem{theo}{Theorem}
\newtheorem{lem}{Lemma}
\newtheorem{defin}{Definition}
\newtheorem{prop}{Proposition}
\newtheorem{ex}{Example}
\newtheorem{alg}{Algorithm}
\newtheorem{cor}{Corollary}
\newtheorem{case}{Case}


% this prints title, author etc. info from above
\begin{frame}
  \titlepage
\end{frame}



\section{Introduction}

\begin{frame}
     \frametitle{Introduction}
\begin{itemize}
\item The Objective is to use ranking algorithms to discover hidden prerequisites in the Math and CS departments.
\item  We have access to data which contains X,000 students, their GPA's and the order in which they have taken their classes.
\item  We will be using the rank centrality and serial rank algorithms.
\item By contrasting the different orderings of classes taken for high gpa students vs. low gpa students, we should be able to find courses that should precede other courses.
\end{itemize}
\end{frame}



\section{Related Work}
\subsection{SerialRank}
\begin{frame}
     \frametitle{Related Work}
\begin{itemize}
\item For SerialRank, we use a matrix C of size n x n of pairwise comparisons, defined as follows :
\begin{equation}
C_{ij} = \left\{
	\begin{array}{rl}
 1 & \text{ for i is ranked higher than j (or i=j)} 	\\
 &  \\
 0 & \text{ for i and j are tied or have no available comparison }	\\
 &  \\
 -1 & \text{ for j is ranked higher than i} 
     \end{array}
   \right.
\label{myExpValue}
\end{equation}
The pairwise similarity matrix is constructed as follows

$$S_{ij}^{match} = \sum_{k=1}^{n}(\frac{1+C_{ik}C_{jk}}{2})$$
Then compute the Laplacian matrix $L_{S} = diag(S1) - S$ and finally computing and sorting the Fiedler vector of S will give the ranking. 

\end{itemize}
\end{frame}

\subsection{Rank-Centrality}

\begin{frame}
     \frametitle{Related Work}
\begin{itemize}
\item For Rank-Centrality, we ???

\end{itemize}
\end{frame}

 


\section{Our work}    \label{sec:ourWork}

\begin{frame}
     \frametitle{ Our work}
\end{frame}



\section{Numerical experiments on synthetic data}  \label{sec:NumExpSyn}

\begin{frame}
\frametitle{Numerical experiments on synthetic data}
\begin{itemize}
\item xyz
\item xyz
\end{itemize}
\end{frame}


\section{Numerical experiments on real data}   \label{sec:NumExpReal}

\begin{frame}
\frametitle{Numerical experiments on real data 1}
\begin{itemize}
\item xyz
\item xyz
\end{itemize}
\end{frame}

\begin{frame}
\frametitle{Numerical experiments on real data 2}
\begin{itemize}
\item xyz
\item xyz
\end{itemize}
\end{frame}




\section{Summary and conclusion}  \label{sec:conclusion}

\begin{frame}
\frametitle{Summary of our work}
\begin{itemize}
\item xyz
\item xyz
\end{itemize}
\end{frame}





\end{document}













\begin{frame}
     \frametitle{ x }
\begin{itemize}
\item 
\item 
\end{itemize}
\end{frame}




\begin{frame}
     \frametitle{ x }
\begin{itemize}
\item 
\item 
\end{itemize}
\end{frame}




\begin{frame}
     \frametitle{ x }
\begin{itemize}
\item 
\item 
\end{itemize}
\end{frame}





