\documentclass{beamer}
\usecolortheme{dolphin}
%\usetheme{Warsaw}

\usepackage{amsmath,amssymb,amsrefs}
\usepackage{amscd}
\usepackage{latexsym}
\usepackage{graphics}
\usepackage{subfigure}
\usepackage{placeins}
\usepackage{booktabs}
\usepackage{multirow}
\usepackage{times}  % fonts are up to you
\usepackage[authoryear]{natbib}
\def\newblock{\hskip .11em plus .33em minus .07em} %this line is used to
% fix the bug of natbib that ``\newblock'' undefined.
\usepackage{array}

\usepackage{lipsum}

 \setbeamersize{text margin left=11pt,text margin right=8pt}



\newcommand\mydef{\mathrel{\overset{\makebox[0pt]{\mbox{\normalfont\tiny\sffamily def}}}{=}}}

\usepackage{soul}



\newif\ifpaper

% \papertrue % or
\paperfalse


\usepackage{graphicx}
\usepackage{nicefrac}
\usepackage{lscape}
\usepackage{times}
% \usepackage[dvips,colorlinks=true,linkcolor=red,filecolor=green,citecolor=red]{hyperref}
%\usepackage[]{amsmath,amssymb,epsfig}
% \usepackage{subfigure}
\usepackage{mathtools}%
\usepackage{algorithm}
\usepackage{algorithmic}
\usepackage{wrapfig}





\newcommand{\mb}[1]{\mbox{\boldmath$#1$}}

\setbeamertemplate{headline}{\scriptsize{\vspace*{0.1cm}\hspace*{0.1cm}\insertframenumber}}

\title{Project title}
\author{ First~Author, Second~Author\\
\vspace{5mm}
MATH 191 Graphs and Networks, UCLA}
% \institute[]{ Applied Mathematics, UCLA
% Applied and Computational Mathematics, \\
% Princeton University
% }
%\address{PACM Graduate Student Seminar}
% \date{ Harvard University \\ February 24, 2014}

\date{ \today}

% note: do NOT include a \maketitle line; also note that this title
% material goes BEFORE the \begin{document}

%%%%% have this if you'd like a recurring outline
\AtBeginSection[]  % "Beamer, do the following at the start of every section"
{
\begin{frame}<beamer>
  \frametitle{Outline} % make a frame titled "Outline"
  \tableofcontents[currentsection]  % show TOC and highlight current section
\end{frame}
}


\begin{document}

\newtheorem{theo}{Theorem}
\newtheorem{lem}{Lemma}
\newtheorem{defin}{Definition}
\newtheorem{prop}{Proposition}
\newtheorem{ex}{Example}
\newtheorem{alg}{Algorithm}
\newtheorem{cor}{Corollary}
\newtheorem{case}{Case}


% this prints title, author etc. info from above
\begin{frame}
  \titlepage
\end{frame}



\section{Introduction}

\begin{frame}
     \frametitle{Introduction}
A list of items:
\begin{itemize}
\item item 1
\item  item 2
\item  item 3
\end{itemize}
\end{frame}



\section{Related Work}

\begin{frame}
     \frametitle{Related Work}
A list of items:
\begin{itemize}
\item item 1
\item  item 2
\item  item 3
\end{itemize}
\end{frame}




 


\section{Our work}    \label{sec:ourWork}

\begin{frame}
     \frametitle{ Our work}
\begin{figure}[h]
\begin{center}
\includegraphics[width=0.76\columnwidth]{Directed_Core_Per}
\end{center}
\caption{A Figure}
\label{fig:DirCorPer}
\end{figure}

\end{frame}



\section{Numerical experiments on synthetic data}  \label{sec:NumExpSyn}

\begin{frame}
\frametitle{Numerical experiments on synthetic data}
\begin{itemize}
\item xyz
\item xyz
\end{itemize}
\end{frame}


\section{Numerical experiments on real data}   \label{sec:NumExpReal}

\begin{frame}
\frametitle{Numerical experiments on real data 1}
\begin{itemize}
\item xyz
\item xyz
\end{itemize}
\end{frame}

\begin{frame}
\frametitle{Numerical experiments on real data 2}
\begin{itemize}
\item xyz
\item xyz
\end{itemize}
\end{frame}




\section{Summary and conclusion}  \label{sec:conclusion}

\begin{frame}
\frametitle{Summary of our work}
\begin{itemize}
\item xyz
\item xyz
\end{itemize}
\end{frame}





\end{document}













\begin{frame}
     \frametitle{ x }
\begin{itemize}
\item 
\item 
\end{itemize}
\end{frame}




\begin{frame}
     \frametitle{ x }
\begin{itemize}
\item 
\item 
\end{itemize}
\end{frame}




\begin{frame}
     \frametitle{ x }
\begin{itemize}
\item 
\item 
\end{itemize}
\end{frame}





